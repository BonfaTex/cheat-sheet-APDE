%!TEX root = ../main.tex

% =================================================
% =================================================

\FSIsection{Helmholtz-Weyl scomposition}

% =================================================
% =================================================

We consider vectorial fields $\bm{u}:\Omega\subseteq\RR^n\to\RR^n$. Capital bold letters denote vectorial functional spaces, e.g. $\mathbf{L}^2(\Omega)=\left[ L^2(\Omega) \right]^n$ or $\mathbf{H}^1(\Omega)= \left[ H^1(\Omega) \right]^n$.

\smallskip

Let $\Omega$ be omitted. We can exploit what we already know, such as
\begin{itemize}

\item[$\triangleright$] $\Lv^2$ is Hilbert with $\norm{\bm{u}}_{\Lv^2}^2=\displaystyle\int_\Omega \abs{\bm{u}}^2=\sum_{i=1}^n \norm{u_i}^2_{L^2}$

\item[$\triangleright$] $\Hv^1$ is Hilbert with
\begin{equation*}
(\bm{u},\bm{v})_{\Hv^1}=(\bm{u},\bm{v})_{\Lv^2}+(\nabla\bm u,\nabla \bm v)_{\Lv^2}= \int_\Omega \Big( \nabla\bm u : \nabla \bm v + \bm u \bm v \Big)
\end{equation*}
where $\nabla\bm u:\nabla\bm v$ stands for the \emph{Euclidean scalar product} of the Jacobian matrices of $\bm u$ and $\bm v$. 

\item[$\triangleright$] $\Hv_0^1\coloneq\overline{\bm \Dc}^{\,\Hv^1}\!$ is Hilbert, and via \emph{Poincaré's inequality}
\begin{equation*}
(\bm u,\bm v)_{\Hv^1}\approx(\bm u,\bm v)_{\Hv^1_0}=(\nabla\bm u,\nabla \bm v)_{\Lv^2}=\int_\Omega \nabla\bm u : \nabla \bm v
\end{equation*}
where symbol $\approx$ denotes \emph{equivalent norms}.

\item[$\triangleright$] $\Ev\coloneq \left\{ \bm u\in \mathbf{L}^2;\ \nabla \cdot \bm u \in L^2 \right\}\equiv \Lv^2_{\text{div}}$ is Hilbert with
\begin{equation*}
(\bm u,\bm v)_\Ev=(\bm u,\bm v)_{\Lv^2}+(\nabla \cdot \bm u,\nabla \cdot\bm v)_{L^2} 
\end{equation*}

Since $\nabla\bm u\in \Lv^2 \impliesnotimplied \nabla\cdot\bm u\in L^2$ (only if $n=1$), then $\Hv^1 \subsetneqq \Ev \subset \Lv^2$. 

We remark that when we say $\nabla\cdot\bm u\in L^2$ we mean in weak (distributional) sense. Indeed, given $\bm u\in\Lv^2$ its weak divergence $\nabla\cdot \bm u\in H^{-1}$ is well defined as
\begin{equation*}
\dualH{\nabla\cdot\bm u,\ \varphi}{-1}{1}{0}=-\int_\Omega \bm u\cdot \nabla \varphi\qquad \forall\, \varphi \in H^1_0\ (\in\Dc)
\end{equation*}

If it happens that $\nabla\cdot \bm u$ is not only $H^{\text{-}1}\!$ but also $L^2$, then $\bm u\in \Ev$ \\ (\emph{"$\bm u$ in $\Lv^2$ with weak divergence in $L^2$, then $\bm u$ in $\Ev$" sounds similar to the older "$u$ in $L^2$ with weak derivative $u'$ in $L^2$, then $u$ in $H^1$"}).

Moreover, for $\bm u\in\Ev$ the normal component of its trace $\gamma_{_\nu}(\bm u)\in H^{\text{-}1/2}$ is well defined and satisfy the \textbf{Generalized Gass-Green formula} (even if $\Omega$ is only Lipschitz):
\begin{equation}
\label{eq:GG}
\int_\Omega \bm u\cdot \nabla \varphi+\int_\Omega \big(\nabla\cdot\bm u\big)\, \varphi = \dualT{\gamma_{_\nu}\bm u,\ \gamma_{_0}\varphi}{-1/2}{1/2}\qquad \forall \,\varphi\in H^1
\end{equation}

Hence, in some way $\Ev$ is an intermediate space: in $\Lv^2$ there are no traces, in $\Ev$ only normal traces, in $\Hv^1$ full traces. One can go through the def. of $\Lv^2_{\text{curl}}$ to find out that there only tangent traces are defined and, via \emph{Friedrich's inequality}, $\Lv^2_{\text{div}}\oplus\Lv^2_{\text{curl}}=\Hv^1$.

\end{itemize}

Let us introduce the following new spaces:
\begin{itemize}
\item[$\triangleright$] $\bm\Vc \coloneq \left\{ \bm u\in\bm\Dc;\ \nabla\cdot\bm u=0 \text{ in }\Omega \right\}$

\item[$\triangleright$] $\Vv\coloneq \overline{\bm\Vc}^{\Hv^1} \!= \left\{ \bm u\in\mathbf{H}_0^1;\ \nabla\cdot\bm u=0\text{ in }\Omega\right\}\equiv\Hv^1_{0,\sigma}$, which is a closed subspace of $\Hv^1_0$ and hence, it is Hilbert if endowed with
\begin{equation*}
(\bm u,\bm v)_\Vv=(\nabla \bm u,\nabla \bm v)_{\Lv^2} 
\end{equation*}

Obviously, $\Vv\subset\Ev$ since free-div. $\impliesnotimplied$ $L^2$-div.

\item[$\triangleright$] $\Gv_1\coloneq \overline{\bm\Vc}^{\Lv^2} \!= \left\{ \bm u\in\mathbf{L}^2;\ \nabla\cdot\bm u=0,\ \gamma_{_\nu} \bm u=0 \right\}\subset\Ev$ 

$\mathbf{G}_2=\left\{ \bm u\in\mathbf{L}^2;\ \nabla\cdot\bm u=0,\ \exists\,g\in H^1\text{ s.t. }\bm u=\nabla g \right\}\subset\Ev$ 

$\mathbf{G}_3=\left\{\bm u\in\mathbf{L}^2;\ \exists\,g\in H^1_0\text{ s.t. }\bm u=\nabla g \right\}\centernot\subset\Ev$

We remark that in $\Gv_2$, the incompressibility constraint makes the scalar function $g$ harmonic:
\begin{equation*}
0=\nabla\cdot\bm u=\nabla\cdot\nabla g=\nabla^2 g=\Delta g
\end{equation*}
Instead, if you add the constraint in $\Gv_3$, you obtain $g\equiv 0$.

\item[$\triangleright$] $\Ev_0\coloneq \ker(\gamma_{_\nu})=\left\{ \bm u\in\Ev;\ \gamma_{_\nu} \bm u=0 \right\}\subset \Ev $
\end{itemize}
\vspace{-0.5em}
\rule{0.495\textwidth}{0.2pt}\smallskip

All these definitions lead to the following properties:
\vspace{-0.5em}
\begin{gather*}
\bm\Vc \overset{\delta}{\subset} \Vv \overset{\delta}{\subset} \Gv_1\subset\Ev_0, \qquad \Gv_1\oplus\Gv_2\subset\Ev, \qquad \Gv_2\cap \Ev_0=\{\bm 0\},\\
\Gv_3 \cap \Ev = \left\{ \bm u \in \Lv^2;\ \exists\,g\in H^2\cap H^1_0 \text{ s.t. }\bm u=\nabla g \right\}, \\
\Gv_3 \cap \Ev_0 = \left\{ \bm u \in \Lv^2;\ \exists\,g\in H^2_0 \text{ s.t. }\bm u=\nabla g \right\}.
\end{gather*}

\textbf{\color{lavender(floral)}Proof.} See whiteboard \texttt{FSI-1} u.u

\smallskip

Instead, in dimension $n=1$ there hold
\begin{gather*}
\Ev\equiv H^1,\quad \bm\Vc=\{0\}\ \Longrightarrow\ \Vv=\Gv_1=\{0\},\\ \Gv_2=\RR,\quad \Gv_3=\big\{ u\in L^2;\ \underbracket[0.5pt]{\textstyle\int_\Omega u=0}_{\scriptstyle\text{zero average}}\big\}\equiv L^2_0.
\end{gather*}

\vspace{-0.5em}
\rule{0.495\textwidth}{0.2pt}\smallskip

\textbf{Thm(H-W scomposition).} $\Gv_1,\ \Gv_2,\ \Gv_3$ are mutually orthogonal (w.r.t. the $\Lv^2$ scalar product), and $\mathbf{L}^2=\mathbf{G}_{1}\oplus\mathbf{G}_{2}\oplus\mathbf{G}_{3}$. In other \emph{words}: 
\begin{gather*}
\mathbf{G}_{i}\perp\mathbf{G}_{j}\ \ \forall\,i\neq j,\qquad \forall\, \bm f \in \Lv^2\ \exists!\,(\bm f_1,\bm f_2,\bm f_3)\in\Gv_1\!\times\!\Gv_2\!\times\!\Gv_3 \\
\qquad\qquad\qquad\qquad\text{ s.t. }\bm f= \bm f_1+\bm f_2+\bm f_3.
\end{gather*}

\textbf{\color{lavender(floral)}Proof.} See whiteboard \texttt{FSI-2} u.u 

\noindent\rlap{\rule[1.5ex]{0.495\textwidth}{.2pt}}
%\rule{0.495\textwidth}{0.2pt}

\newcolumn

% =================================================
% =================================================

\FSIsection{Stokes Problem (Stationary Linearized NS)}

% =================================================
% =================================================

Let us consider the following problem:
\begin{subequations}
    \label{Stokes-problem}
    \begin{align}[left=\empheqlbrace]
    -\eta\Delta \bm u +\nabla p = \bm f &\qquad\text{ in }\Omega  \label{eq:sp1} \\
    \nabla\cdot \bm u=0 &\qquad\text{ in }\Omega \label{eq:sp2} \\
    \bm u=0 &\qquad\text{ on }\partial\Omega \label{eq:sp3}
    \end{align}
\end{subequations}

where $\bm u=(u_1,\dots,u_n)$ (you may think $n=3$) is the velocity vector, $\Delta\bm u$ is the vector laplacian, $p$ is the scalar pressure of the fluid, $\eta>0$ its dynamic viscosity, and $\bm f\in\mathbf{L}^2$ is a given vector function representing the density of the external force applied to the fluid.

Condition \eqref{eq:sp2} represents the \emph{incompressibility constraint}, while \eqref{eq:sp3} the \emph{non-slip boundary condition}, i.e. the case in which the fluid is adherent to the boundary of $\Omega$. One can use different bcs as long as there holds the GG formula \eqref{eq:GG} for $\bm u$ and $1$, which acts as a compatibility condition:
\begin{equation*}
\int_\Omega \bm u\ \cancel{\nabla 1} + \int_\Omega \big(\cancel{\nabla\cdot \bm u}\big)\, 1= \sca{\gamma_{_\nu}\bm u, 1}\quad\leadsto\quad \sca{\gamma_{_\nu}\bm u, 1}=0
\end{equation*}

In system \eqref{Stokes-problem}, the unknowns we look for are $\bm u$ and $p$. Notiche that \eqref{eq:sp1}+\eqref{eq:sp2} are $n+1$ scalar equations in the $n+1$ unknowns $u_1,\dots,u_n,p$ while \eqref{eq:sp3} gives just $n$ bcs. The reason is that $p$ appears in $\eqref{eq:sp1}$ only through its gradient, therefore the pressure is determined up to an additive constant (if $\overline{p}$ is a solution, $\overline{p}+k$ is a solution too). 

\noindent\rlap{\rule[1.5ex]{0.495\textwidth}{.2pt}}

\vspace{-0.5em}

\FSIsubsection{Variational Formulation}

It is natural to think about $\bm u\in\Vv$ and $p\in H^1\!/\RR$, then \eqref{eq:sp1} becomes
\begin{equation*}
\int_\Omega -\eta\Delta\bm u\ \bm v +\int_\Omega \nabla p\ \bm v =\int_\Omega \bm f\ \bm v\qquad \forall\, \bm v\in\Vv
\end{equation*}

The integrations by parts yield
\begin{equation*}
\eta \int_\Omega \nabla\bm u:\nabla\bm v-\int_{\partial\Omega}\eta\,\frac{\partial \bm u}{\partial \nu}\ \cancel{\bm v}+\int_{\partial\Omega} \cancel{\bm v} \cdot \nu\ p -\int_\Omega \cancel{\nabla\cdot\bm v}\ p = \int_\Omega \bm f\ \bm v
\end{equation*}

Then the \textbf{weak formulation} is 
\begin{equation*}
\big(\forall\,\bm f\in\Lv^2\big)\text{ Find }\bm u\in \Vv\text{ :}\quad \eta\int_\Omega \nabla \bm u : \nabla \bm v=\int_\Omega \bm f\ \bm v\quad \forall\,\bm v\in\mathbf{V}\tag{WF1}
\end{equation*}

By LM (\textbf{\color{lavender(floral)}Proof:} See wb \texttt{FSI-3} u.u), $\big(\forall\,\bm f\in\Lv^2\big)$ $\exists\,!\,\bm u \in \mathbf{V}$ satisfying (WF1).

\noindent\rlap{\rule[1.5ex]{0.495\textwidth}{.2pt}}\vspace{-0.5em}

Now we are at the lowest level, but we'd like to recover the pressure in the formulation. Applying the blackbox called \emph{elliptic regularity theory}, 
\begin{equation*}
\big(\forall\,\bm f\in\Lv^2\big)\ \exists\,!\,\bm u \in \mathbf{V}\cap\Hv^2 \textit{ satisfying }\text{(WF1)}
\end{equation*}

Then undo the int. by parts over $\Delta\bm u$, and since $\Vv$ is dense in $\Gv_1$:
\begin{equation*}
\big(\forall\,\bm f\in\Lv^2\big)\ \exists\,!\,\bm u \in \mathbf{V}\cap\Hv^2 \textit{ satisfying } \int_\Omega \Big( \eta\Delta \bm u+\bm f\Big)\bm v=0\ \ \forall\,\bm v\in\Gv_1
\end{equation*}

This means that $\big( \eta\Delta \bm u+\bm f\big)\equiv \nabla p \perp \Gv_1$ in $\Lv^2$ $\Longrightarrow$ $\nabla p \in \mathbf{G}_2\oplus \mathbf{G_3}$, indeed a posteriori one can see the $p$ disappearing without integrate by parts:
\begin{equation*}
\int_\Omega {\nabla p}\ {\bm v} = 0\qquad\text{'cause}\qquad \nabla p\in \Gv_2\oplus\Gv_3,\ \bm v\in \Gv_1
\end{equation*}

Hence, (by definition of $\Gv_2\oplus\Gv_3$) $\exists\,!\,p\in H^1\!/\RR$ and we can state that
\begin{gather*}
\big(\forall\,\bm f\in\Lv^2\big)\ \exists\,!\,(\bm u,p) \in \big(\mathbf{V}\cap\Hv^2\big)\times\big(H^1\!/\RR\big) \\ 
\textit{satisfying } \eqref{eq:sp1}+\eqref{eq:sp2}\textit{ a.e. and } \eqref{eq:sp3} \textit{ in the sense of traces}
\end{gather*}

which works as definition of \textbf{strong solution} (+ more reg. $\Rightarrow$ \textbf{classical}).

\medskip

\textbf{Rmk:} $\underbracket[0.5pt]{\Delta\bm u\in \Gv_1\oplus\Gv_2}_{\textbf{\color{lavender(floral)}Proof.} \text{ wb }\texttt{FSI-3}}, \nabla p\in \Gv_2\oplus\Gv_3$ $\Rightarrow$ vel. and p. interact only on $\Gv_2$.

\smallskip

There is another way to recover the pressure based on Banach Closed Range thm and Nečas ineq. \textbf{\color{lavender(floral)}Proof:} See whiteboard \texttt{FSI-4} u.u

\noindent\rlap{\rule[1.5ex]{0.495\textwidth}{.2pt}}

\vspace{-0.5em}

\FSIsubsection{Generalized Stokes Problem}

Let $\Omega\subset\RR^{n\geq 2}$ open bdd connected with Lipschitz boundary. Set $\bm f\in\Hv^{-1}(\Omega)$, $\bm v_*\in \Hv^{1/2}(\partial\Omega)$, $g\in L^2(\Omega)$. If the compatibility condition
\begin{equation*}
\sca{\gamma_{_\nu}\bm v_*,1}=\int_\Omega g
\end{equation*}

is satisfied, then $\exists\,!$ weak solution $(\bm u,p)\in \Hv^1(\Omega)\times L^2_0(\Omega)$ of
\begin{align*}[left=\empheqlbrace]
-\eta\Delta \bm u +\nabla p = \bm f &\qquad\text{ in }\Omega \nonumber \\
\nabla\cdot \bm u=g &\qquad\text{ in }\Omega \nonumber \\
\bm u=\bm v_* &\qquad\text{ on }\partial\Omega \nonumber
\end{align*}

(here the spaces are \emph{weaker} bc we're not looking for strong solutions)

\smallskip

Moreover, $\exists\,C>0$ (depending only on $\Omega$, $n$) s.t.
\begin{equation*}
\norm{\bm u}_{\Hv^1}+\norm{p}_{L^2}\leq C \Big( \norm{\bm f}_{\Hv^{\text{-}1}}+\norm{\bm v_*}_{\Hv^{1/2}} + \norm{g}_{L^2} \Big)
\end{equation*}

\textbf{\color{lavender(floral)}Proof.} It's a very istructive proof, based on Bogovskii's operator and Nečas inequality. See whiteboard \texttt{FSI-5} u.u

\noindent\rlap{\rule[1.5ex]{0.495\textwidth}{.2pt}}

\newpage



